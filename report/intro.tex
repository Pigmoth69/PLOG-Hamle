\chapter{Introdu\c{c}\~ao}

Neste relat\'orio estudaremos a gera\c{c}\~ao de solu\c{c}\~oes com base na programa\c{c}\~ao em l\'ogica com restri\c{c}\~oes, demonstrando os resultados obtidos e comparando os tempos de execu\c{c}\~ao para tabuleiros de diferentes dimens\~oes.\\
Relativamente ao problema abordado, o Hamle consiste num jogo de tabuleiro quadrangular (6x6), com dez pe\c{c}as pretas posicionadas em locais espec\'ificos. Por sua vez, cada uma destas pe\c{c}as apresentar\'a um n\'umero correspondente ao deslocamento que dever\'a efetuar em qualquer umas das quatro dire\c{c}\~oes poss\'iveis. Ap\'os o deslocamento, nenhuma pe\c{c}a preta poder\'a ser adjacente a outra, assim como todas as c\'elulas vazias dever\~ao estar interligadas, ou seja, nao poder\~a existir dois ou mais grupos distintos de c\'elulas vazias.\\
Quanto \`a estrutura\c{c}\~ao do relat\'orio, orientar-nos-emos por uma an\'alise detalhada \`as restri\c{c}\~oes impostas, descrevendo de seguida o c\'odigo implementado, estudando os resultados apresentados em terminal e discutindo os valores e solu\c{c}\~oes obtidas.